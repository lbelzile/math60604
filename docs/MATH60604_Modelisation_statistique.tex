% Options for packages loaded elsewhere
\PassOptionsToPackage{unicode}{hyperref}
\PassOptionsToPackage{hyphens}{url}
%
\documentclass[
]{book}
\usepackage{lmodern}
\usepackage{amssymb,amsmath}
\usepackage{ifxetex,ifluatex}
\ifnum 0\ifxetex 1\fi\ifluatex 1\fi=0 % if pdftex
  \usepackage[T1]{fontenc}
  \usepackage[utf8]{inputenc}
  \usepackage{textcomp} % provide euro and other symbols
\else % if luatex or xetex
  \usepackage{unicode-math}
  \defaultfontfeatures{Scale=MatchLowercase}
  \defaultfontfeatures[\rmfamily]{Ligatures=TeX,Scale=1}
\fi
% Use upquote if available, for straight quotes in verbatim environments
\IfFileExists{upquote.sty}{\usepackage{upquote}}{}
\IfFileExists{microtype.sty}{% use microtype if available
  \usepackage[]{microtype}
  \UseMicrotypeSet[protrusion]{basicmath} % disable protrusion for tt fonts
}{}
\makeatletter
\@ifundefined{KOMAClassName}{% if non-KOMA class
  \IfFileExists{parskip.sty}{%
    \usepackage{parskip}
  }{% else
    \setlength{\parindent}{0pt}
    \setlength{\parskip}{6pt plus 2pt minus 1pt}}
}{% if KOMA class
  \KOMAoptions{parskip=half}}
\makeatother
\usepackage{xcolor}
\IfFileExists{xurl.sty}{\usepackage{xurl}}{} % add URL line breaks if available
\IfFileExists{bookmark.sty}{\usepackage{bookmark}}{\usepackage{hyperref}}
\hypersetup{
  pdftitle={Modélisation statistique},
  hidelinks,
  pdfcreator={LaTeX via pandoc}}
\urlstyle{same} % disable monospaced font for URLs
\usepackage{longtable,booktabs}
% Correct order of tables after \paragraph or \subparagraph
\usepackage{etoolbox}
\makeatletter
\patchcmd\longtable{\par}{\if@noskipsec\mbox{}\fi\par}{}{}
\makeatother
% Allow footnotes in longtable head/foot
\IfFileExists{footnotehyper.sty}{\usepackage{footnotehyper}}{\usepackage{footnote}}
\makesavenoteenv{longtable}
\usepackage{graphicx,grffile}
\makeatletter
\def\maxwidth{\ifdim\Gin@nat@width>\linewidth\linewidth\else\Gin@nat@width\fi}
\def\maxheight{\ifdim\Gin@nat@height>\textheight\textheight\else\Gin@nat@height\fi}
\makeatother
% Scale images if necessary, so that they will not overflow the page
% margins by default, and it is still possible to overwrite the defaults
% using explicit options in \includegraphics[width, height, ...]{}
\setkeys{Gin}{width=\maxwidth,height=\maxheight,keepaspectratio}
% Set default figure placement to htbp
\makeatletter
\def\fps@figure{htbp}
\makeatother
\setlength{\emergencystretch}{3em} % prevent overfull lines
\providecommand{\tightlist}{%
  \setlength{\itemsep}{0pt}\setlength{\parskip}{0pt}}
\setcounter{secnumdepth}{5}
% \usepackage{amsmath,amssymb,mathtools}
\usepackage{enumerate}
\usepackage{geometry}
\geometry{hmargin=1.2in}
\usepackage[mathscr]{eucal}
\DeclareMathAlphabet{\mathcrl}{U}{rsfs}{m}{n}
\usepackage{utopia}
\DeclareMathAlphabet{\mathcal}{OMS}{cmsy}{m}{n}
\usepackage{booktabs}
\usepackage{amsthm}
\makeatletter
\def\thm@space@setup{%
  \thm@preskip=8pt plus 2pt minus 4pt
  \thm@postskip=\thm@preskip
}
\makeatother

\usepackage{framed,color}
\definecolor{shadecolor}{RGB}{248,248,248}

\renewcommand{\textfraction}{0.05}
\renewcommand{\topfraction}{0.8}
\renewcommand{\bottomfraction}{0.8}
\renewcommand{\floatpagefraction}{0.75}

\let\oldhref\href
\renewcommand{\href}[2]{#2\footnote{\url{#1}}}

\ifxetex
  \usepackage{letltxmacro}
  \setlength{\XeTeXLinkMargin}{1pt}
  \LetLtxMacro\SavedIncludeGraphics\includegraphics
  \def\includegraphics#1#{% #1 catches optional stuff (star/opt. arg.)
    \IncludeGraphicsAux{#1}%
  }%
  \newcommand*{\IncludeGraphicsAux}[2]{%
    \XeTeXLinkBox{%
      \SavedIncludeGraphics#1{#2}%
    }%
  }%
\fi

\makeatletter
\newenvironment{kframe}{%
\medskip{}
\setlength{\fboxsep}{.8em}
 \def\at@end@of@kframe{}%
 \ifinner\ifhmode%
  \def\at@end@of@kframe{\end{minipage}}%
  \begin{minipage}{\columnwidth}%
 \fi\fi%
 \def\FrameCommand##1{\hskip\@totalleftmargin \hskip-\fboxsep
 \colorbox{shadecolor}{##1}\hskip-\fboxsep
     % There is no \\@totalrightmargin, so:
     \hskip-\linewidth \hskip-\@totalleftmargin \hskip\columnwidth}%
 \MakeFramed {\advance\hsize-\width
   \@totalleftmargin\z@ \linewidth\hsize
   \@setminipage}}%
 {\par\unskip\endMakeFramed%
 \at@end@of@kframe}
\makeatother

\makeatletter
\@ifundefined{Shaded}{
}{\renewenvironment{Shaded}{\begin{kframe}}{\end{kframe}}}
\makeatother

\newenvironment{rmdblock}[1]
  {
  \begin{itemize}
  \renewcommand{\labelitemi}{
    \raisebox{-.7\height}[0pt][0pt]{
      {\setkeys{Gin}{width=3em,keepaspectratio}\includegraphics{images/#1}}
    }
  }
  \setlength{\fboxsep}{1em}
  \begin{kframe}
  \item
  }
  {
  \end{kframe}
  \end{itemize}
  }
\newenvironment{rmdnote}
  {\begin{rmdblock}{note}}
  {\end{rmdblock}}
\newenvironment{rmdcaution}
  {\begin{rmdblock}{caution}}
  {\end{rmdblock}}
\newenvironment{rmdimportant}
  {\begin{rmdblock}{important}}
  {\end{rmdblock}}
\newenvironment{rmdtip}
  {\begin{rmdblock}{tip}}
  {\end{rmdblock}}
\newenvironment{rmdwarning}
  {\begin{rmdblock}{warning}}
  {\end{rmdblock}}
\usepackage[]{natbib}
\bibliographystyle{apalike2}

\title{Modélisation statistique}
\author{}
\date{\vspace{-2.5em}}

\begin{document}
\maketitle

\let\href\oldhref

{
\setcounter{tocdepth}{1}
\tableofcontents
}
\hypertarget{remarques}{%
\chapter*{Remarques}\label{remarques}}
\addcontentsline{toc}{chapter}{Remarques}

Ces notes sont l'oeuvre de Léo Belzile (HEC Montréal) et sont mis à disposition sous la \href{https://creativecommons.org/licenses/by-nc-sa/4.0/legalcode.fr}{Licence publique Creative Commons Attribution - Utilisation non commerciale - Partage dans les mêmes conditions 4.0 International} et ont été compilé le 16 juin 2020.

Bien que les diapositives illustrent l'implémentation des techniques statistiques et des modèles à l'aide de \textbf{SAS}, ces notes présentent le pendant \textbf{R}: visitez \href{https://cran.r-project.org/}{le site web du projet \textbf{R}} pour télécharger le logiciel. L'interface graphique la plus populaire (et celle que je vous recommande) est \href{https://www.rstudio.com/products/rstudio/download/}{RStudio Desktop}.

\newcommand{\bs}[1]{\boldsymbol{#1}}
\newcommand{\Hmat}{\mathbf{H}}
\newcommand{\Mmat}{\mathbf{M}}
\newcommand{\mX}{\mathbf{X}}
\newcommand{\bX}{{\mathbf{X}}}
\newcommand{\bx}{{\mathbf{x}}}
\newcommand{\by}{{\boldsymbol{y}}}
\newcommand{\bY}{{\boldsymbol{Y}}}
\newcommand{\eps}{\varepsilon}
\newcommand{\beps}{\boldsymbol{\varepsilon}}
\newcommand{\bbeta}{\boldsymbol{\beta}}
\newcommand{\hbb}{\hat{\boldsymbol{\beta}}}
\newcommand{\limni}{\lim_{n \ra \infty}}
\newcommand{\Sp}{\mathscr{S}}
\newcommand{\Hy}{\mathscr{H}}
\newcommand{\E}[2][]{{\mathsf E}_{#1}\left(#2\right)}
\newcommand{\Va}[2][]{{\mathsf{Var}_{#1}}\left(#2\right)}
\newcommand{\I}[1]{{\mathbf 1}_{#1}}

\hypertarget{intro}{%
\chapter{Introduction à l'inférence statistique}\label{intro}}

\hypertarget{regression-lineaire}{%
\chapter{Régression linéaire}\label{regression-lineaire}}

\hypertarget{modeles-lineaires-generalises}{%
\chapter{Modèles linéaires généralisés}\label{modeles-lineaires-generalises}}

\hypertarget{donnees-correlees-longitudinales}{%
\chapter{Données corrélées et longitudinales}\label{donnees-correlees-longitudinales}}

\hypertarget{modeles-lineaires-mixtes}{%
\chapter{Modèles linéaires mixtes}\label{modeles-lineaires-mixtes}}

\hypertarget{survie}{%
\chapter{Analyse de survie}\label{survie}}

\hypertarget{vraisemblance}{%
\chapter{Inférence basée sur la vraisemblance}\label{vraisemblance}}

\hypertarget{appendix-annexe}{%
\appendix}


\hypertarget{r}{%
\chapter*{\texorpdfstring{\textbf{R}}{R}}\label{r}}
\addcontentsline{toc}{chapter}{\textbf{R}}

  \bibliography{book.bib,packages.bib}

\end{document}
